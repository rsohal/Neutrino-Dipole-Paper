\begin{section}{Introduction}
  \label{sec:introduction}

The properties of neutrinos are of great interest in modern 
physics and in particular, much work has been done in constraining 
their masses. Neutrinos are now known to be massive and oscillation 
experiments have accurately measured the squared mass differences 
between the three species. Additionally, modern cosmological 
observations have placed constraints on the sum of the neutrino 
masses.

\par However, current techniques are unable to deduce the 
individual masses of the neutrinos and place no constraints 
on the mass hierarchy. Recently, however, a technique was 
proposed to constrain individual neutrino masses using the 
relative flow between neutrinos and dark matter \cite{bib:Zhu2013, 
bib:Zhu2014}. Due to their small masses, neutrinos will 
travel large distances and have their trajectories only
 perturbed through gravitational interaction. In particular, 
through interaction with dark matter, neutrinos can be focused 
into wakes. This overdensity introduces a dipole distortion 
in the neutrino-CDM cross-correlation function. Although the 
relative velocity fields themselves are not directly observable, 
they can be deduced from the galaxy density field. This is possible as the 
coherence of the relative velocity over large distances implies 
that linear perturbation theory is applicable, as shown in 
\cite{bib:Inman}. As this dipole is present solely due to the neutrino-CDM 
gravitational interaction, it provides a signal for the masses 
of the neutrinos.

\par To compute the dipole correlation function, it is pertinent 
to use a neutrino velocity field as the dipole depends on the 
neutrino mass and velocity. In particular, one may consider the 
velocity fields corresponding to neutrinos of varying initial 
energy. Given that neutrinos with lower energies are more likely 
to cluster, one expects that their velocity fields should yield 
a larger dipole.  

\par In this paper, we present a method of calculating the 
dipole correlation function in an N-body simulation in which
neutrinos and dark matter are implemented as particles. We consider 
the impact of using the velocity fields of neutrinos with 
varying initial energies. In particular, we wish to determine which
velocity field maximizes the dipole. We first discuss our implementation of 
the N-body code \cpm \cite{bib:HarnoisDeraps2013}, the calculation 
of the velocity fields, and the calculation of the dipole correlation 
function. We then present and discuss the results from our simulations.

\end{section}

