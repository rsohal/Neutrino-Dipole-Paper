\begin{section}{Discussion}
  \label{sec:discussion}

  We have calculated the dipole correlation function of neutrinos 
grouped according to their initial velocities using velocity fields
calculated from a momentum method and reconstructed from CDM and halo
density fields. We found that the density and velocity power spectra
of the neutrino bins are consistent with expectations; specifically,
neutrinos with smaller initial velocities have greater density and
velocity power at all scales.
 
\par The differences between the dipole correlation functions 
of the velocity bins were not as great, however. With the 
exception of the dipole of the $0-5214\kms$ velocity bin 
calculated using the CDM reconstruction method, there was no significant 
difference between the dipoles of the velocity bins for $r>5\mpch$. 
This was to be expected as it has been shown previously that 
\cite{bib:Inman} that reconstruction via linear theory provides an 
accurate estimate of the actual velocity field. In particular, the 
deviation at small scales of the dipole correlation functions produced
using linear theory suggests that the errors result from non-linear
evolution (e.g. due to clustering of neutrinos). At smaller scales, 
the larger velocity bins had greater dipole correlation functions. 
This suggests that in computing the dipole from observations, it 
may be advantageous to use the velocity fields of neutrinos with 
greater initial velocities in order to increase the signal at small 
scales.  
 
\end{section}
